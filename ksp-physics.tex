% KSP Physics Documentation by Mhoram
% This work is licensed under the Creative Commons
% Attribution-NonCommercial-ShareAlike 4.0 International License.
\documentclass[11pt]{report}

\usepackage[utf8]{inputenc}
\usepackage[a4paper]{geometry}
\usepackage[pdfstartview=Fit,pdfpagelayout=SinglePage,breaklinks=true]{hyperref}
\usepackage{breakurl}
\usepackage{multicol}
\usepackage{amsfonts}
\usepackage{amsmath}
\usepackage[final]{listings}
\usepackage{makeidx}
\makeindex
\usepackage[nottoc]{tocbibind}

\newcommand{\oa}[1]{\overrightarrow{#1}}
\newcommand{\F}[1]{\oa{F_{#1}}}
\newcommand{\Pos}{\oa{P}}
\newcommand{\Vel}{\oa{V}}
\newcommand{\absvec}[1]{\left|\left|{#1}\right|\right|}
\newcommand{\dddvec}[3]{\left(\begin{smallmatrix}{#1}\\{#2}\\{#3}\end{smallmatrix}\right)}

\DeclareMathOperator{\atan2}{atan2}
\DeclareMathOperator{\Lat}{Lat}
\DeclareMathOperator{\Lon}{Lon}

\immediate\write18{git describe --tags | sed -e "s/v//" -e "s/^\string\([0-9]\string\+[.][0-9]\string\+\string\)$/\string\1.0/" -e "s/-/./" -e "s/-.\string\+//" > .tagversion} % $

\setcounter{chapter}{-1}

\begin{document}

\title{Documentation of KSP Physics\\\small{Version \input{.tagversion}}\\\vspace{1 em}\url{https://github.com/mhoram-kerbin/ksp-physics-documentation}\\\url{http://forum.kerbalspaceprogram.com/threads/93426-Physics-of-KSP}}
\author {Mhoram Kerbin \href{mailto:mhoram.kerbin@gmx.de}{\nolinkurl{mhoram.kerbin@gmx.de}}}

\maketitle

\begin{abstract}

  This document summarizes the stock physics of Kerbal Space Program
  \cite{KSP}. It should be valid for versions 0.21 to 0.24.2. Most
  equations are also applicable to real life situations - if you ever
  happen to stumble into one.

\end{abstract}

\tableofcontents

\chapter{Preface}

\section{Introduction}

During my attempts to use PSOPT as a method for generating ideal
ascent trajectories for rockets, I needed to implement the physics of
KSP in C/C++. So I had to gather all the physics related information
again after my Ascent Optimizer Perl-project \cite{PAO}. Since they
are available at quite different places, this document is intended to
summarize them in a single central place.

\section{Document Structure}

Part \ref{InnerWorkings} describes how the Physics Engine of KSP
works. Part \ref{InGamePhysics} describes how some in game physical
effects can be calculated. Appendix \ref{RealWorldDifferences}
contains a desription of the differences between the real world and
KSP-Physics.

\section{Relevance for Physics-Changing Addons}

FAR \cite{FAR} and NEAR \cite{NEAR} implement a completely different
atmosphere and only the non-atmospheric descriptions of this text can
be applied to them.

While Principia \cite{principia} intends to change the orbital
mechanics completely it remains to be seen how much of the rest of
this text can be applied to it.

\part{Inner Workings of KSP}\label{InnerWorkings}

\chapter{Conventions}

We denote by $\Pos$, $\Vel$, $d$ and $M$ the \index{Position Vector}
position vector, \index{Velocity Vector} velocity vector, drag
coefficient and the mass of the ship respectively.

$\Pos_X, \Pos_Y \textrm{ and } \Pos_Z$ denote the X, Y and Z
components of the vector $\Pos$.

The Z-Axis points from the planets core to the northpole.

Distance denotes the distance to the planets center and altitude the
distance to the surface sea level.

For Vector-Operations $\oa{A} \cdot \oa{B}$ denotes the scalar product
\cite{ScalarProduct} and $\oa{A} \times \oa{B}$ denotes the cross
product \cite{CrossProduct}.

\chapter{Physical Constants}

\section{Global Constants}

The \index{conversion factor} conversion factor between pressure and
density is according to \cite{Atmo}.

\begin{align}
  CF &:= 1.2230948554874 \frac{kg}{m^3\cdot atm}
\end{align}

The \index{gravitational constant} gravitational constant $G$ is according to \cite{ACB}.
\begin{align}
  G &:= 6.674E-11 N\left(\frac{m}{kg}\right)^2
\end{align}

\section{Planets}

Planets are defined by the following parameters

\begin{itemize}
\item Name
\item \index{Mass of Planet} Mass of Planet $PMass^{(Name)}$ in kg
\item \index{Planetary Radius} Planetary Radius $PRadius^{(Name)}$ in m
\item \index{Scale Height} Scale Height $PSH^{(Name)}$ in m
\item \index{Pressure at Sealevel} Pressure at Sealevel $p_0^{(Name)}$ in atm
\item \index{Rotation Period} Rotation Period $RP^{(Name)}$ in s
\item \index{Sphere of Influence} Radius of Sphere of Influence $Soi^{(Name)}$ in m \footnote{It should be noted that the Sphere of Influence is actually calculated
as described in \cite{SphereOfInfluence}.}
\end{itemize}

The KSP Wiki \cite{Wiki} contains information about each celestial
body.

\subsection{Kerbin}

According to the KSP Wiki for Kerbin the values are:

\begin{align*}
  PMass^{(Kerbin)} &:= 5.2915793E22 kg\\
  PRadius^{(Kerbin)} &:= 600000 m\\
  PSH^{(Kerbin)} &:= 5000 m\\
  p_0^{(Kerbin)} &:= 1 atm\\
  RP^{(Kerbin)} &:= 21600 s\\
  Soi^{(Kerbin)} &:= 84159286 m\\
\end{align*}

\chapter{Atmosphere}

\section{Atmospheric Formula}

Now we will show how several atmospheric parameters are calculated
based on \cite{Atmo}.
\begin{align}
  \index{Atmospheric Height} \textrm{AtmosphericHeight}\nonumber\\
  AH^{(Name)} &:= -ln\left(10^{-6}\right)\cdot PSH^{(Name)}\label{AtmosphericHeightEquation}\\
  \index{Pressure} \textrm{Pressure at altitude }&\nonumber\\
  p(alt) &:=
  \left\{
      \begin{array}{l l}
        p_0^{(Name)} \cdot exp\left({\frac{-alt}{PSH^{(Name)}}}\right) & \textrm{if } alt < AH^{(Name)}\\
        0 &\textrm{if } alt \geq AH^{(Name)}
      \end{array}
    \right.\\
    \index{Density} \textrm{Density at altitude}\nonumber\\
    \rho(alt) &:= CF \cdot p(alt)
\end{align}


\section{Lift}

The best description of \index{Lift} lift can be found in the ``How to
calculate lift?'' thread \cite{Lift}

\chapter {Surface}

\section{Latitude and Longitude conversion}

From a Position Vector $\Pos$ we can calculate \index{Latitude}
latitude and \index{Longitude} longitude by the following formula:

\begin{align}
  \Lon(\Pos) &:= \arctan\left(\frac{\Pos_Y}{\Pos_X}\right)\\
  \Lat(\Pos) &:= \arctan\left(\frac{\Pos_Z}{\sqrt{\Pos_X^2+\Pos_Y^2}}\right)
\end{align}

{\bf Be advised that this happens to be only true if the inertial X is
  aligned with body X at t = 0. That's not the case for most KSP
  planets and moons.} \cite{LatLongNote}

\section{Surface Velocity}

Due to the rotation of the planet, the \index{Surface Velocity}
surface velocity is different from the orbital speed.

The difference between the two speed vectors $DiffV(\Pos)$ is the
projected planets rotation to the position $\Pos$ of the rocket.

We can convert orbital $\Vel$ to surface velocity $\oa{SV}$
vectors with the following equation if we know the position $\Pos$.

\begin{align}
  DiffV(\Pos) &:= \absvec{\Pos} \cdot \frac{2 \pi}{RP^{(Name)}} \cdot \cos\left(\Lat\left(\Pos\right)\right)\cdot\left(\begin{smallmatrix}-\sin(\Lon(\Pos)\\\cos(\Lon(\Pos))\\0\end{smallmatrix}\right)\nonumber\\
  \oa{SV} &= \Vel - DiffV(\Pos)
\end{align}

\chapter{Forces}

There are three forces that effect the flightpath of a rocket, namely
Drag, Gravity and Thrust. This section details them.

\section{Drag}

\index{Drag} Drag is the force applied to rocket based on the movement
through the atmosphere and it is directed in an opposite direction to
the movement.

It should be notet that since the atmosphere rotates with the planet,
the surface velocity is the relevant velocity and not the orbital
velocity.

\subsection{Drag Coefficient}

In KSP $d$ is the \index{Drag Coefficient} drag coefficient of the
rocket. It is a mass-based average of the drag coefficients of all
parts of the rocket and is dimensionless.

It indicates how much it is decellerated by means of air friction.

Usually it is near to $0.2$.

Consider a rocket consisting of a Mk1 Cockpit (dry mass of 1.25 ton,
drag coefficient of 0.1), a FL-T800 Fuel Tank (total mass of 4.5 ton,
drag coefficient of 0.2) and a LV-909 Liquid Fuel Engine (total mass
of 0.5 ton, drag coefficient of 0.2). For this rocket the drag
coefficient is:
\begin{align}
  d &= \frac{1.25 \cdot 0.1 + 4.5 \cdot 0.2 + 0.5 \cdot 0.2}{ 1.25 + 4.5 + 0.5 }\nonumber\\
  &= \frac{1.125}{6.25}\nonumber\\
  &= 0.18\nonumber
\end{align}

\subsection{Drag Force}\label{DragForce}

Since the drag force $\F{D}$ is directly proportional to the
cross-sectional area $A$ of the rocket and this value is approximated
in KSP by $A = 0.008 \frac{m^2}{kg} \cdot M$ we get

\begin{align}
  \F{D} &:= - 0.5 \cdot \rho(alt)\cdot \absvec{\oa{SV}}^2\cdot d \cdot 0.008\frac{m^2}{kg} \cdot M \cdot \frac{\oa{SV}}{\absvec{\oa{SV}}} \nonumber\\
  &= - 0.5 \cdot \rho(alt)\cdot d \cdot 0.008\frac{m^2}{kg} \cdot M \cdot \absvec{\oa{SV}} \cdot \oa{SV}
\end{align}

\section{Gravity}

The \index{Gravity} gravity force $\F{G}$ is directed towards the
planets core and gets lower with increasing distance.

\begin{align}
  \mu^{(Name)} = LocalGravityParameter^{(Name)} &:= G \cdot PMass^{(Name)} \nonumber\\
  LocalGravity(distance) &:= \frac{LocalGravityParameter}{distance^2}\nonumber\\
  \F{G} &:= - M \cdot \mu^{(Name)}\cdot\frac{\Pos}{\absvec{\Pos}^3}
\end{align}

\section{Engines}

\subsection{Rocket Engines}

The \index{ISP} Specific impulse $I_{SP}$ (in seconds) describes the
engine efficiency and is parameterized for an engine by ISP at
sealevel and ISP in vacuum.

The ISP is linear in the pressure $p$ and cut off at $1atm$
(citation needed).  We can calculate the ISP based on the pressure
$p$ by
\begin{align}
  np(p) &:= \min(1 atm, p)\\
  ISP(p) &:= ISP_{1atm} \cdot np(p) + ISP_{VAC} \cdot (1-np(p))
\end{align}

According to \cite{ECF} the conversion factor $g_0$ is not
$9.81\frac{m}{s^2}$, but
\begin{align}
  g_0 &\approx 9.82\frac{m}{s^2}
\end{align}

So we get a fuel consumption $\dot{M}$ (change of mass) \cite{SPI} of:
\begin{align}
\dot{M} & := \frac{\absvec{\oa{F_T}}}{ISP \cdot g_0}
\end{align}

\subsection{Jet Engines}

Jet Engines are described in a nonlinear way. Details can be found in
the ``Fuel consumption as a function of atmospheric pressure`` thread
\cite{JetEngines}

\subsection{ISP Calculation}

A ship with multiple engines that have different ISP-profiles has an
averaged ISP that is based on the ISP of all running engines.

The calculation of the ships ISP is described on the wiki
\cite{MulEng}.

A ship with $n$ running engines, where each engine $i$ has a thrust of
$F_{T_i}$ and a specific impulse of $I_{SP_i}$ has a total ISP of:

\begin{align}
  I_{SP} := \frac{\sum_i{F_{T_i}}}{\sum_i{\frac{F_{T_i}}{I_{SP_i}}}}
\end{align}

\subsection{Engine Thrust}

The total \index{Thrust} ship thrust $\F{T}$ with $n$ running engines
with thrusts $\F{T_i}$ is

\begin{align}
  \F{T} & := \sum_{i}\F{T_i}\label{enginethrust}
\end{align}

This equation assumes that the engines do not induce a rotation onto
the rocket. If this is however the case then the total thrust is lower
than calculated in \eqref{enginethrust}.

It should also be noted that this equation is a vectoraddition,
meaning that the orientation of the engines is important.

\chapter{Acceleration}

The rocket is subect to the three forces drag $\F{D}$, gravity $\F{G}$
and thrust $\F{T}$.  The \index{Acceleration} acceleration as a vector
$\oa{a}$ that the rocket experiences based on these forces is:

\begin{align}
  \oa{a} &:= M^{-1} \cdot(\F{D} + \F{G} + \F{T})
\end{align}

\chapter{Ship Rotation}

There are several forces that induce a rotation on a ship.

\begin{itemize}
\item \index{Torque} Torque from SAS modules and most command pods
\item Engines that are not aligned with the center of mass
\item \index{Gimbal} Gimbal of engines
\item Drag on rocketparts
\item Colisions with other objects
\end{itemize}

At the time of this writing there is to my knowledge no
community-based comparison of the quantities of these effects
available.

The best description of Torque can be found in the ``If something has
an SAS torque of `20', what does that actually mean?'' thread
\cite{torque}.

\chapter{Coordinate Conversion}

An \index{Orbit} Orbit can be described either by \index{Cartesian
  Coordinates} cartesian coordinates of a position vector $\Pos$ and a
velocity vector $\Vel$ or by \index{Kepler Orbit} Kepler Elements
\cite{Kepler} that consist of

\begin{itemize}
\item \index{Semi-major axis} Semi-major axis ($a$)
\item \index{Eccentricity} Eccentricity ($e$)
\item \index{Inclination} Inclination ($i$)
\item \index{Ascending node} Longitude of the ascending node ($\Omega$)
\item \index{Argument of periapsis} Argument of periapsis ($\omega$)
\item True Anomaly ($\nu$), Mean Anomaly ($M$) or Eccentric Anomaly ($E$)
\end{itemize}

In KSP, Kepler Elements are used for objects that are on rails and
Cartesian Coordinates for the ship you currently steer and others in
its vicinity.

This section contains a description of the conversion between Kepler
and Cartesian coordinates.

\section{Orbit-Position Conversions} \label{OrbitPositionConversions}

For a Kepler Orbit there are three aequivalent parameters that
describe the position of an object on the orbit.

\begin{itemize}
\item \index{True Anomaly} True Anomaly ($\nu$) \cite{TrueA}
\item \index{Mean Anomaly} Mean Anomaly ($M$) \cite{MeanA}
\item \index{Eccentric Anomaly} Eccentric Anomaly ($E$) \cite{EccentricA}
\end{itemize}

All of them are useful for different purposes. This section deals with
converting them.

Note that we use $M$ for both, Rocketmass and Mean Anomaly. The
context usually indicates which of both is the intended one.

There are other equivalent parameterizations possible, like for
example time since Periapsis.

\subsection{True Anomaly to Eccentric Anomaly} \label{ta2ea}

The Eccentric Anomaly $E$ can be calculated from the True Anomaly $\nu$
and Eccentricity $e$ by the formula
\begin{align}
  \tan(E) & = \frac{\sqrt{1-e^2}\cdot \sin(\nu)}{e+\cos(\nu)}\nonumber\\
  E & = \atan2\left(\sqrt{1-e^2}\cdot \sin(\nu), e+\cos(\nu)\right)
\end{align}

\subsection{Mean Anomaly to Eccentric Anomaly} \label{ma2ea}

We calculate the eccentric Anomaly $E$ by the formula $M = E - e\cdot
\sin(E)$. Since this equation does not have a closed-form solution for
$E$, we use the Newton-Raphson method \cite{NeRaMe} to approximate the
solution.

In order to start the iteration, we need some initial values.
\begin{align}
  MaxError & := 10^{-13}\nonumber\\
  target & := \left\{
    \begin{array}{l l}
      \pi & \textrm{if } e > 0.8\\
      M & \textrm{otherwise}
    \end{array}
  \right.\nonumber\\
  error & := target - e\cdot\sin(target) - M\nonumber
\end{align}

After that we apply the following steps \eqref{NeRaBegin} to \eqref{NeRaEnd} until $|error| < MaxError$ or
the procedure has found no solution within a specified number of runs.
\begin{align}
  prev & := target\label{NeRaBegin}\\
  target & := prev - \frac{error}{1 - e\cdot \cos(prev)}\\
  error & := target - e \cdot\sin(target) - M\label{NeRaEnd}
\end{align}

The resulting Eccentric anomaly is $E := target$.

\subsection{Eccentric Anomaly to True Anomaly} \label{ea2ta}

We get the True Anomaly by the formula

\begin{align}
\tan\left(\frac{\nu}{2}\right) & = \sqrt{\frac{1+e}{1-e}} \cdot \tan\left(\frac{E}{2}\right)\nonumber\\
\nu & = 2\cdot\atan2\left(\sqrt{1+e}\cdot \sin\left(\frac{E}{2}\right), \sqrt{1-e}\cdot \cos\left(\frac{E}{2}\right)\right)
\end{align}

\subsection{Eccentric Anomaly to Mean Anomaly} \label{ea2ma}

We get the Mean Anomaly from the Eccentric Anomaly and the
Eccentricity by the formula

\begin{align}
M & = E - e\cdot \sin(E)
\end{align}

\subsection{True Anomaly to Mean Anomaly}

Use the methods in sections \ref{ta2ea} and \ref{ea2ma}.

\subsection{Mean Anomaly to True Anomaly}

Use the methods in sections \ref{ma2ea} and \ref{ea2ta}.

\section{Cartesian to Kepler}

We can convert cartesian to kepler coordinates by the following
equations that are explained in \cite{RSCK}, where $\mu^{(Body)}$
denotes the gravitational Parameter of the planet, $\oa{h}$ the
\index{Orbital Momentum} orbital momentum vector and $\oa{e}$ the
eccentricity vector.

\begin{align}
  a &:= \left(2\cdot\absvec{\Pos}^{-1} - \frac{\absvec{\Vel}^2}{\mu^{(Body)}}\cdot \right)^{-1}\\
  \oa{h} &:= \Pos \times \Vel\nonumber\\
  \oa{e} &:= \frac{\Vel\times \oa{h}}{\mu^{(Body)}} - \absvec{\Pos}\nonumber\\
  e &:= \absvec{\oa{e}}\\
  i &:= \arccos\left(\frac{\oa{h}_Z}{\absvec{\oa{h}}} \right)\\
  \oa{n} & := \dddvec{0}{0}{1} \times \oa{h} = \dddvec{-\oa{h}_Y}{\oa{h}_X}{0}\nonumber\\
  \Omega &:= \left\{
    \begin{array}{l l}
      0 & \textrm{if } \absvec{\oa{n}} = 0\\
      \arccos\left(\dddvec{1}{0}{0}\cdot\frac{\oa{n}}{\absvec{\oa{n}}} \right) & \textrm{if } \oa{n}_Y = \oa{h}_X >= 0\\
      2\pi - \arccos\left(\dddvec{1}{0}{0}\cdot\frac{\oa{n}}{\absvec{\oa{n}}} \right) & \textrm{otherwise}
    \end{array}
    \right.\\
  \omega & := \left\{
    \begin{array}{l l}
      0 & \textrm{if } \absvec{\oa{n}} = 0\\
      \arccos\left(\frac{\oa{n}\cdot\oa{e}}{\absvec{\oa{n}}\cdot\absvec{\oa{e}}} \right) & \textrm{if } \oa{e}_Z >= 0\\
      2\pi - \arccos\left(\frac{\oa{n}\cdot\oa{e}}{\absvec{\oa{n}}\cdot\absvec{\oa{e}}}\right) & \textrm{otherwise}
    \end{array}
    \right.\\
    \nu &:= \left\{
    \begin{array}{l l}
      \arccos\left(\frac{\Pos\cdot\oa{e}}{\absvec{\Pos}\cdot\absvec{\oa{e}}}\right) & \textrm{if } \Pos\cdot\Vel >= 0\\
      2\pi - \arccos\left(\frac{\Pos\cdot\oa{e}}{\absvec{\Pos}\cdot\absvec{\oa{e}}}\right) & \textrm{otherwise}
    \end{array}
    \right.
\end{align}

\section{Kepler to Cartesian}

We can also convert kepler to cartesian coordinates by the following
equations that are explained in \cite{RSKC}, where $\mu^{(Body)}$
denotes the Gravitational Parameter of the planet.

Use the formulas in section \ref{OrbitPositionConversions} to get the
True Anomaly $\nu$ and Eccentric Anomaly $E$. In this calculation
$\oa{o}$ and $\dot{\oa{o}}$ denote the position and velocity vector
within the \index{Orbital Frame of Reference} orbital frame of reference.


\begin{align}
  \absvec{\oa{o}} & := a(1-e \cdot \cos(E))\nonumber\\
  \oa{o} & := \absvec{\oa{o}} \dddvec{\cos(\nu)}{\sin(\nu)}{0}\nonumber\\
  \dot{\oa{o}} & := \frac{\sqrt{\mu^{(Body)} a}}{\absvec{\oa{o}}} \dddvec{-\sin(E)}{\sqrt{1-e^2}\cos(E)}{0}\nonumber\\
  \Pos & := \dddvec{\oa{o}_X(\cos(\omega)\cos(\Omega)-sin\omega\cos(i)\sin(\Omega)) - \oa{o}_Y(\sin(\omega)\cos(\Omega)+\cos(\omega)\cos(i)\sin(\Omega))}{\oa{o}_X(\cos(\omega)\sin(\Omega)+sin\omega\cos(i)\cos(\Omega)) - \oa{o}_Y(\cos(\omega)\cos(i)\cos(\Omega)-\sin(\omega)\sin(\Omega))}{\oa{o}_X(\sin(\omega)\sin(i))+\oa{o}_Y(cos\omega\sin(i))}\\
  \Vel & := \dddvec{\dot{\oa{o}}_X(\cos(\omega)\cos(\Omega)-sin\omega\cos(i)\sin(\Omega)) - \dot{\oa{o}}_Y(\sin(\omega)\cos(\Omega)+\cos(\omega)\cos(i)\sin(\Omega))}{\dot{\oa{o}}_X(\cos(\omega)\sin(\Omega)+sin\omega\cos(i)\cos(\Omega)) - \dot{\oa{o}}_Y(\cos(\omega)\cos(i)\cos(\Omega)-\sin(\omega)\sin(\Omega))}{\dot{\oa{o}}_X(\sin(\omega)\sin(i))+\dot{\oa{o}}_Y(cos\omega\sin(i))}
\end{align}

\part{Physics in the Game}\label{InGamePhysics}

\chapter{Orbital Mechanics}

\section{Orbital Calculations}

We can calculate the \index{Apoapsis} apoapsis $r_A$ and
\index{Periapsis} periapsis $r_P$ (in the meaning distance to the
Planets center) as

\begin{align}
r_A & := a \cdot (1 + e)\\
r_P & := a \cdot (1 - e)
\end{align}

Conversely we can calculate the eccentricity and semi-major as

\begin{align}
  a & := \frac{1}{2}\cdot(r_A + r_P)\\
  e & := \frac{r_A}{a} - 1\\
  e & := 1 - \frac{r_P}{a}
\end{align}

The velocity $V$ of an object that orbits a Body and has a distance
$r$ to the Planets center is according to the Vis-viva equation
\cite{VisViva}

\begin{align}
V & := \sqrt{\mu^{(Body)} \cdot \left(\frac{2}{r} - \frac{1}{a}\right)}
\end{align}

It should be noted that the speed reaches it's maximum at Periapsis
and minimum at Apoapsis and that the disctance $r$ lies within the
bounds

$$
r_P \leq r \leq r_A
$$

In the case of a circular orbit, the speed is equal at all points on
the orbit.

The \index{Orbital Period} orbital period (with the meaning of
\index{Siderial Period} sidereal period \cite{SIDPER}) around a Body
is

\begin{align}
SiderealPeriod := 2\pi\sqrt{\frac{a^3}{\mu^{(Body)}}}
\end{align}

\section{Orbital Darkness Time}

A ship orbiting a Body will spend some time in the shadow of the
body. \index{Orbital Darknes Time}

The maximal possible time $T_d$ in darkness can be calculated as
described in \cite{DarknessTime}.

\begin{align}
  b & := \sqrt{r_A \cdot r_P} \textrm{ the semi-minor axis}\index{Semi-minor axis}\\
  l & := 2\frac{r_A\cdot r_P}{r_A + r_P} \textrm{ the semi-latus rectum} \index{Semi-latus rectum}\\
  h & := \sqrt{l\mu^{(Name)}} \textrm{ the specific angular momentum}\index{Specific angular momentum}\\
  T_d &:= \frac{2ab}{h}\left(\arcsin\left(\frac{PRadius^{(Name)}}{b}\right)+\frac{e\cdot PRadius^{(Name)}}{b}\right)
\end{align}

\part{Appendices}

\chapter{Differences to the real world}\label{RealWorldDifferences}

\section{Atmosphere}

In the real world the atmosphere does not stop suddenly at an
altitude.\index{Atmospheric Height}

In KSP every Planetary Body with an atmosphere has an altitude above
of which there is no atmosphere and this altitude can be calculated as
described in equation \ref{AtmosphericHeightEquation}.

The density of the atmosphere also depends on the temperature and
humidity \cite{RealDensity} but KSP does not model these.

\section{Drag}

In the real world Drag is directly proportional to the cross-sectional
area $A$ of the ship.  In KSP this value is approximated by the ships
mass $M$ as described in section \ref{DragForce}.

With the Add-Ons FAR \cite{FAR} and NEAR \cite{NEAR} an effort is made
to implement a more realistic atmosphere into KSP.

\section{Gravitational Model}

In the real world every Planetary Body exercises gravitation to all
other bodies.  In KSP a simplified model is used that is based on
Patched Conics \index{Patched Conics} \cite{PatchedConics}.

This simplification states that the space is divided into areas where
only the gravitation of a single planetary body is in effect.  These
areas are mostly spheric, have the same center as the planetary body
and they can contain other such areas, if the planetary body has other
planetary bodies orbiting it.

The name ``Sphere of Influence'' \index{Sphere of Influence} can refer
either to the area itself or to it's radius. This radius can be
calculated according to the formula in \cite{SphereOfInfluence}.

The Add-On Principia \cite{principia} (currently under development)
intends to implement a more realistic
N-Body-Physics. \index{N-Body-Physics}

\section{Engine Thrust}

In the real world Rockete-engines do not have a constant thrust.
% TODO

%\section{Index}
\printindex

\begin{thebibliography}{Kerbin}

\bibitem{KSP}
  \url{http://kerbalspaceprogram.com/}

\bibitem{PAO}
  \url{https://github.com/mhoram-kerbin/ascent-optimizer}

\bibitem{Atmo}
  \url{http://wiki.kerbalspaceprogram.com/wiki/Atmosphere}

\bibitem{ACB}
  \url{http://wiki.kerbalspaceprogram.com/wiki/API:CelestialBody}

\bibitem{Wiki}
  \url{http://wiki.kerbalspaceprogram.com/wiki/Main_Page}

\bibitem{Kerbin}
  \url{http://wiki.kerbalspaceprogram.com/wiki/Kerbin}

\bibitem{ECF}
  \url{http://wiki.kerbalspaceprogram.com/wiki/Specific_impulse#Conversion_factor}

\bibitem{Kepler}
  \url{http://en.wikipedia.org/wiki/Kepler_orbit}

\bibitem{RSCK}
  \url{https://downloads.rene-schwarz.com/dc/category/20}

\bibitem{RSKC}
  \url{https://downloads.rene-schwarz.com/dc/category/19}

\bibitem{SPI}
  \url{http://wiki.kerbalspaceprogram.com/wiki/Specific_impulse#Formula}

\bibitem{SIDPER}
  \url{http://en.wikipedia.org/wiki/Sidereal_period}

\bibitem{MulEng}
  \url{http://wiki.kerbalspaceprogram.com/wiki/Specific_impulse#Multiple_engines}

\bibitem{Lift}
  \url{http://forum.kerbalspaceprogram.com/threads/29788-How-to-calculate-lift}

\bibitem{JetEngines}
  \url{http://forum.kerbalspaceprogram.com/threads/67606-Fuel-consumption-as-a-function-of-atmospheric-pressure}

\bibitem{TrueA}
  \url{http://en.wikipedia.org/wiki/True_anomaly}

\bibitem{MeanA}
  \url{http://en.wikipedia.org/wiki/Mean_anomaly}

\bibitem{EccentricA}
  \url{http://en.wikipedia.org/wiki/Eccentric_anomaly}

\bibitem{NeRaMe}
  \url{http://en.wikipedia.org/wiki/Newton\%27s_method}

\bibitem{FAR}
  \url{http://forum.kerbalspaceprogram.com/threads/20451-Ferram-Aerospace-Research}

\bibitem{NEAR}
  \url{http://forum.kerbalspaceprogram.com/threads/86419-NEAR-A-Simpler-Aerodynamics-Model}

\bibitem{torque}
  \url{http://forum.kerbalspaceprogram.com/threads/75916-If-something-has-an-SAS-torque-of-20-what-does-that-actually-mean}

\bibitem{principia}
  \url{http://forum.kerbalspaceprogram.com/threads/68502-Principia-N-Body-Gravitation}

\bibitem{ScalarProduct}
  \url{http://en.wikipedia.org/wiki/Dot_product}

\bibitem{CrossProduct}
  \url{http://en.wikipedia.org/wiki/Cross_product}

\bibitem{VisViva}
  \url{http://en.wikipedia.org/wiki/Vis-viva_equation}

\bibitem{DarknessTime}
  \url{http://wiki.kerbalspaceprogram.com/wiki/Orbit_darkness_time}

\bibitem{PatchedConics}
  \url{http://en.wikipedia.org/wiki/Patched_conic_approximation}

\bibitem{SphereOfInfluence}
  \url{http://wiki.kerbalspaceprogram.com/wiki/Sphere_of_influence}

\bibitem{RealDensity}
  \url{http://en.wikipedia.org/wiki/Density_of_air}

\bibitem{LatLongNote}
\url{http://forum.kerbalspaceprogram.com/threads/18230?p=1433324#post1433324}

\end{thebibliography}
\end{document}

% Local Variables:
% ispell-local-dictionary:"american";
% End:

